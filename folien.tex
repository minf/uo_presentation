\documentclass[ngerman,compress,hyperref={bookmarks}]{beamer}
%\usetheme{Frankfurt}
%\usetheme{boxes}
%\usetheme{Malmoe}
\usetheme{Antibes}
\useoutertheme{infolines}
%\useoutertheme{split}
\usepackage[utf8x]{inputenc}
%\usepackage[nolist]{acronym}

\usepackage{colortbl}
\definecolor{dunkelgrau}{rgb}{0.8, 0.8, 0.8}

\usepackage{wasysym}

\logo{\includegraphics[height=1cm]{logoHAW}}
\usepackage{graphicx}
%\usepackage[%
%	bibstyle=authoryear,%
%	citestyle=authoryear,%
%	bibencoding=utf8,%
%	bibtex8=true,%
%	sorting=nyt,%
%	sortcites=true,%
%	maxnames=2,%
%	babel=other,%
%	block=space,%
%	backref=false,%
%	natbib=true,%
%	hyperref=true,%
%]{biblatex}
%\bibhang1em
%\usepackage[style=authortitle-icomp]{biblatex}
%\bibliography{routing_atlas}

\setbeamertemplate{bibliography entry title}{}
\setbeamertemplate{bibliography entry location}{}

\title{Lernen und Wissen in Unternehmen}
\subtitle{Vortrag: Unternehmensorientierung}
\subject{}
\author{Benjamin Vetter \and Andreas Krohn}
\institute[HAW]{Hochschule für Angewandte Wissenschaften Hamburg}
\date[WS 2011/12]{11. Januar 2012}

\begin{document}
\frame[plain]{\titlepage}

\part{Intro}
\section*{Agenda}
\begin{frame}{Agenda} \setcounter{tocdepth}{1} \tableofcontents[part=1] \setcounter{tocdepth}{3} \end{frame}

\part{Hauptteil}
\section{Wissen im Unternehmen}
\begin{frame}{Wissen im Unternehmen}
\begin{itemize}
 \item Trend zur Wissensarbeit
 \item ``Immaterielle'' Wertschöpfung
 \item Metapher: Fabrik $\rightarrow$ Küche
 \item Berücksichtigung und Organisation von Wissen benötigt
\end{itemize}
\end{frame}

\subsection{Definition: Wissen}
\begin{frame}[allowframebreaks]
\frametitle{Definition: Wissen}
\begin{description}
 \item[Zeichen] Symbole, kontextfrei
 \item[Daten] Verkettung von Zeichen, frei von einem Verwendungszweck
 \item[Informationen] Daten im Kontext eines Problemzusammenhangs
 \item[Wissen] Eine Menge von Informationen, Modellierte Wirklichkeit
\end{description}
\framebreak
\begin{description}
 \item[Explizit] Regeln, Texte
 \item[Implizit] Intuitives Handeln, Wiedererkennen bekannter Situationen
\end{description}
\end{frame}

\subsection{Wissensgenerierung}
\begin{frame}{Wissensgenerierung}
Intern
\begin{itemize}
 \item Forschung und Entwicklung
 \item Anreize zum Wissenstransfer
 \item Technische Infrastruktur
\end{itemize}
Extern
\begin{itemize}
 \item Forschungskooperationen
 \item Lead-User
 \item Lieferanten-Kunden-Netzwerk
\end{itemize}
\end{frame}

\subsection{Wissensmanagement}
\begin{frame}{Wissensmanagement}

\end{frame}
\end{document}
