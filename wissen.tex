
%\frame{\titlepage}

\section{Wissen im Unternehmen}

\frame{
  \frametitle{Agenda}

  \tableofcontents
}

% Bedingt durch Globalisierung, Wettbewerb und Informationstechnik streben Unternehmen danach hochwertige Produkte und Dienstleistungen anzubieten
% Dafür sind die Mitarbeiter und deren Fähigkeiten die wichtigste Resource
% Sie verfügen über ``Wissen''

% Eine Definition des Begriffs ``Wissen'' ist jedoch kaum zu finden bzw. je nach Kontext und Autor verschieden
% Ansätze: 
% - Zugänglichkeit: Implizites und Explizites Wissen
% - Personelle Bindung: Individuelles und Kollektives Wissen
% - Wissen vs. Information vs. Daten [vs. Zeichen]

% Nonaka und Takeuchi: Kulturelle Unterschiede
% - Für westl. Unternehmen ist Wissen explizit. ``Explizites Wissen lässt isch in Worten und Zahlen ausdrücken und problemlos mit Hilfe von Daten, wissenschaflichen Formeln, festgelegten Verfahren oder universellen Prinzipien mitteilen'' [Nonake, Takeuchi, Die Organisation des Wissens]
% - Japan. Unternehmen ist der Stellenwert impliziten Wisses deutlich höher. ``Subjektive Einsichten, Ahnungen und Intuition fallen in diese Wissenskategorie. Darüber hinaus ist das implizite Wissen tief verankert in der Tätigkeit und deer Erfahrung des einzelnen sowie in seinen Idealen, Werten und Gefühlen'' [s.o.]

% Franken: Hauptformen des Wissens in Unternehmen [Franken, Integriertes Knowledge Management]
% (S. 303)
% 1- Strukturiertes Wissen
% 2- Unstrukturiertes, formalisiertes Wissen
% 3- Personelles, individuelles Wissen
% 4- Kollektives Wissen
% Umgang mit 3 & 4 im Unternehmen nicht einfach

% Wozu Wissensmanagement? (S. 304)
% - Kompetenzen der Mitarbeiter kennen, nutzen (und ausbauen)..
%	.. da sonst:
%		- externe Dienstleistungen bezahlt werden, die In-House erledigt werden können
% - Austausch von Informationen innerhalb des Unternehmens..
%	.. da sonst:
%		- eine Abteilung etwas entwickelt, das eine andere Abteilung bereits fertiggestellt hat
% - IMHO Wichtigster Punkt:
% 	Um bei Kündigung, Pension etc. das Wissen im Unternehmen zu halten.
% -> Problemfall implizites Wissen
% 	Wie motiviere ich meine Mitarbeiter, Wissen weiterzugeben und von anderen anzunehmen?
% 		- Ideen sammeln? (Fragerunde?, Erfahrungen?, ..)


\subsection{Einführung}
\frame{
  \frametitle{Einführung}
  \begin{itemize}
    \item Trend zur Wissensarbeit
    \item ``Immaterielle'' Wertschöpfung
    \item Metapher: Fabrik $\rightarrow$ Küche
  \end{itemize}
}

\subsection{Definition: Wissen}
\begin{frame}[allowframebreaks]
\frametitle{Definition: Wissen}
\begin{description}
 \item[Zeichen] Symbole, kontextfrei
 \item[Daten] Verkettung von Zeichen, frei von einem Verwendungszweck
 \item[Informationen] Daten im Kontext eines Problemzusammenhangs
 \item[Wissen] Eine Menge von Informationen, Modellierte Wirklichkeit
\end{description}
\textbf{Beispiel}: Wechselkurse %``1'' ``,'' ``4'' ``9'' $\rightarrow$ ``1,49'' $\rightarrow$ ``1,49 EUR/USD'' $\rightarrow$ Verhalten von Kursen angesichts von Leitzins, Markt, \ldots
\framebreak
\begin{description}
 \item[Explizit] Regeln, Texte
 \item[Implizit] Intuitives Handeln, Wiedererkennen bekannter Situationen
\end{description}
\end{frame}

\subsection{Wissensgenerierung}
\begin{frame}{Wissensgenerierung}
Intern
\begin{itemize}
 \item Forschung und Entwicklung
 \item Anreize zum Wissenstransfer
 \item Technische Infrastruktur
\end{itemize}
Extern
\begin{itemize}
 \item Forschungskooperationen
 \item Lead-User
 \item Lieferanten-Kunden-Netzwerk
\end{itemize}
\end{frame}



